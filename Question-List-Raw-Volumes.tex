%%%%%%%%%%%%%%%%%%%%%%%%%%
%%%%%%%%%%%%%%%%%%%%%%%%%%



%%%%%%%%%%%%%%%%%%%%%%%
%%\tagged{Ans@ShortAns, Type@Compute, Topic@Volume, Sub@Solid, File@0001}{
\begin{sagesilent}
# Define variables and constants/exponents
a=randint(1,10)
b=randint(1,10)

#Define the volume
Ans=1/3*a^2*b
\end{sagesilent}

\latexProblemContent{
\ifVerboseLocation This is Volume Compute Question 0001. \\ \fi
\begin{problem}
Find the volume of the pyramid whose base is a square with sides of length $\sage{a}$ and whose height is $\sage{b}$.

\[\answer{\sage{Ans}}\; \mbox{units}^3\]

\end{problem}}%}
%%%%%%%%%%%%%%%%%%%%%%

%%%%%%%%%%%%%%%%%%%%%%%
%%\tagged{Ans@ShortAns, Type@Compute, Topic@Volume, Sub@Solid, File@0002}{
\begin{sagesilent}
# Define variables and constants/exponents
a=randint(1,10)

#Define the volume
Ans=2*sqrt(3)/3*a^3
\end{sagesilent}

\latexProblemContent{
\ifVerboseLocation This is Volume Compute Question 0002. \\ \fi
\begin{problem}
Find the volume of the solid whose base is a disk of radius $\sage{a}$ and whose cross-sections are equilateral triangles.

\[\answer{\sage{Ans}}\; \mbox{units}^3\]

\end{problem}}%}
%%%%%%%%%%%%%%%%%%%%%%

%%%%%%%%%%%%%%%%%%%%%%%
%%\tagged{Ans@ShortAns, Type@Compute, Topic@Volume, Sub@Solid, File@0003}{
\begin{sagesilent}
# Define variables and constants/exponents
a=randint(1,10)

#Define the volume
Ans=2*pi/3*a^3
\end{sagesilent}

\latexProblemContent{
\ifVerboseLocation This is Volume Compute Question 0003. \\ \fi
\begin{problem}
Find the volume of the solid whose base is a disk of radius $\sage{a}$ and whose cross-sections are semi-circles.

\[\answer{\sage{Ans}}\; \mbox{units}^3\]

\end{problem}}%}
%%%%%%%%%%%%%%%%%%%%%%

%%%%%%%%%%%%%%%%%%%%%%%
%%\tagged{Ans@ShortAns, Type@Compute, Topic@Volume, Sub@Solid, File@0004}{
\begin{sagesilent}
# Define variables and constants/exponents
a=randint(1,10)

#Define the volume
Ans=16/3*a^3
\end{sagesilent}

\latexProblemContent{
\ifVerboseLocation This is Volume Compute Question 0004. \\ \fi
\begin{problem}
Find the volume of the solid whose base is a disk of radius $\sage{a}$ and whose cross-sections are squares.

\[\answer{\sage{Ans}}\; \mbox{units}^3\]

\end{problem}}%}
%%%%%%%%%%%%%%%%%%%%%%

%%%%%%%%%%%%%%%%%%%%%%%
%%\tagged{Ans@ShortAns, Type@Compute, Topic@Volume, Sub@Disk, Sub@Washer, File@0005}{
\begin{sagesilent}
# Define variables and constants/exponents
a=randint(1,10)
b=randint(1,10)
c=randint(1,10)

# Define the function
F=a*sqrt(b*x)

#Define the volume
Ans=pi*a^2*b*c^2/2
\end{sagesilent}

\latexProblemContent{
\ifVerboseLocation This is Volume Compute Question 0005. \\ \fi
\begin{problem}
Find the volume of the solid generated when the are bounded by the curve $y=\sage{F}$, the $x-$axis, and the line $x=\sage{c}$ is revolved about the $x-$axis.

\[\answer{\sage{Ans}}\; \mbox{units}^3\]

\end{problem}}%}
%%%%%%%%%%%%%%%%%%%%%%

%%%%%%%%%%%%%%%%%%%%%%%
%%\tagged{Ans@ShortAns, Type@Compute, Topic@Volume, Sub@Disk, Sub@Washer, File@0006}{
\begin{sagesilent}
# Define variables and constants/exponents
a=randint(1,10)
c=randint(1,10)

# Define the function
F=a*x^(1/3)

#Define the volume
Ans=3*pi*a^2*c^(5/3)/5
\end{sagesilent}

\latexProblemContent{
\ifVerboseLocation This is Volume Compute Question 0006. \\ \fi
\begin{problem}
Find the volume of the solid generated when the are bounded by the curve $y=\sage{F}$, the $x-$axis, and the line $x=\sage{c}$ is revolved about the $x-$axis.

\[\answer{\sage{Ans}}\; \mbox{units}^3\]

\end{problem}}%}
%%%%%%%%%%%%%%%%%%%%%%

%%%%%%%%%%%%%%%%%%%%%%%
%%\tagged{Ans@ShortAns, Type@Compute, Topic@Volume, Sub@Disk, Sub@Washer, File@0007}{
\begin{sagesilent}
# Define variables and constants/exponents
a=randint(1,10)
b=randint(1,10)
c=randint(1,10)

# Define the function
F=a*x^2+b

#Define the volume
Ans=pi*(a^2*c^5/5+2*a*b*c^3/3+b^2*c)
\end{sagesilent}

\latexProblemContent{
\ifVerboseLocation This is Volume Compute Question 0007. \\ \fi
\begin{problem}
Find the volume of the solid generated when the are bounded by the curve $y=\sage{F}$, the $x-$axis, and the line $x=\sage{c}$ is revolved about the $x-$axis.

\[\answer{\sage{Ans}}\; \mbox{units}^3\]

\end{problem}}%}
%%%%%%%%%%%%%%%%%%%%%%

%%%%%%%%%%%%%%%%%%%%%%%
%%\tagged{Ans@ShortAns, Type@Compute, Topic@Volume, Sub@Disk, Sub@Washer, File@0008}{
\begin{sagesilent}
# Define variables and constants/exponents
a=randint(1,10)
b=randint(1,10)
c=randint(1,10)

# Define the function
F=a*(b*x)^(1/2)

#Define the volume
Ans=pi*(a*c^2*(b*c)^(1/2)-(a*b^(5/2)*c^(5/2))/(5*b^2))
\end{sagesilent}

\latexProblemContent{
\ifVerboseLocation This is Volume Compute Question 0008. \\ \fi
\begin{problem}
Find the volume of the solid generated when the are bounded by the curve $y=\sage{F}$, the $x-$axis, and the line $x=\sage{c}$ is revolved about the $y-$axis.

\[\answer{\sage{Ans}}\; \mbox{units}^3\]

\end{problem}}%}
%%%%%%%%%%%%%%%%%%%%%%

%%%%%%%%%%%%%%%%%%%%%%%
%%\tagged{Ans@ShortAns, Type@Compute, Topic@Volume, Sub@Disk, Sub@Washer, File@0009}{
\begin{sagesilent}
# Define variables and constants/exponents
a=randint(1,10)
c=randint(1,10)

# Define the function
F=a*x^(1/3)

#Define the volume
Ans=pi*6*a*c^(7/3)/7
\end{sagesilent}

\latexProblemContent{
\ifVerboseLocation This is Volume Compute Question 0009. \\ \fi
\begin{problem}
Find the volume of the solid generated when the are bounded by the curve $y=\sage{F}$, the $x-$axis, and the line $x=\sage{c}$ is revolved about the $y-$axis.

\[\answer{\sage{Ans}}\; \mbox{units}^3\]

\end{problem}}%}
%%%%%%%%%%%%%%%%%%%%%%

%%%%%%%%%%%%%%%%%%%%%%%
%%\tagged{Ans@ShortAns, Type@Compute, Topic@Volume, Sub@Disk, Sub@Washer, File@0010}{
\begin{sagesilent}
# Define variables and constants/exponents
a=randint(1,10)
c=randint(1,10)

# Define the function
F=a*x^2

#Define the volume
Ans=pi*a*c^4/2
\end{sagesilent}

\latexProblemContent{
\ifVerboseLocation This is Volume Compute Question 0010. \\ \fi
\begin{problem}
Find the volume of the solid generated when the are bounded by the curve $y=\sage{F}$, the $x-$axis, and the line $x=\sage{c}$ is revolved about the $y-$axis.

\[\answer{\sage{Ans}}\; \mbox{units}^3\]

\end{problem}}%}
%%%%%%%%%%%%%%%%%%%%%%




















