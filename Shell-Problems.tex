

%%%%%%%%%%%%%%%%%%%%%%%
%%\tagged{Ans@ShortAns, Type@Compute, Topic@Volume, Sub@Washer, Sub@Shells, File@0011}{
\begin{sagesilent}
# Define variables and constants/exponents
a=randint(1,10)
p=randint(0,2)
q=NonZeroInt(0,2,[p])

#Defining function choices
v=[a*x^3, a*x^2, a*x]
Rout=v[max(p,q)]
Rin=v[min(p,q)]

#Define the volume
Ans=pi*integral((Rout)^2-(Rin)^2, x,0,1)
\end{sagesilent}

\latexProblemContent{
\ifVerboseLocation This is Volume Compute Question 0011. \\ \fi
\begin{problem}
Find the volume of the solid generated when the are bounded by the curve $y=\sage{Rout}$ and $y=\sage{Rin}$ is revolved about the $x-$axis. (Use either washers or shells)

\[\answer{\sage{Ans}}\; \mbox{units}^3\]

\end{problem}}%}
%%%%%%%%%%%%%%%%%%%%%%

%%%%%%%%%%%%%%%%%%%%%%%
%%\tagged{Ans@ShortAns, Type@Compute, Topic@Volume, Sub@Washer, Sub@Shells, File@0012}{
\begin{sagesilent}
var('x,y')
# Define variables and constants/exponents
a=randint(1,10)
p=randint(0,2)
q=NonZeroInt(0,2,[p])

#Defining function choices
v=[a*x^3, a*x^2, a*x]
vflip=[(y/a)^(1/3), (y/a)^(1/2), y/a]
Rout=vflip[min(p,q)]
Rin=vflip[max(p,q)]
rout=v[min(p,q)]
rin=v[max(p,q)]

#Define the volume
Ans=pi*integral((Rout)^2-(Rin)^2, y,0,1)
\end{sagesilent}

\latexProblemContent{
\ifVerboseLocation This is Volume Compute Question 0012. \\ \fi
\begin{problem}
Find the volume of the solid generated when the are bounded by the curve $y=\sage{rout}$ and $y=\sage{rin}$ is revolved about the $x-$axis. (Use either washers or shells)

\[\answer{\sage{Ans}}\; \mbox{units}^3\]

\end{problem}}%}
%%%%%%%%%%%%%%%%%%%%%%

%%%%%%%%%%%%%%%%%%%%%%%
%%\tagged{Ans@ShortAns, Type@Compute, Topic@Volume, Sub@Washer, Sub@Shells, File@0013}{
\begin{sagesilent}
var('x,y')
# Define variables and constants/exponents
c=NonZeroInt(-5,5,[0,1])
p=randint(0,2)
q=NonZeroInt(0,2,[p])

#Defining function choices
v=[x^3, x^2, x]
vmin=v[min(p,q)]
vmax=v[max(p,q)]

#Define the volume
if c>1:
   Ans=pi*integral((c-vmin)^2-(c-vmax)^2, x,0,1)
if c<0:
   Ans=pi*integral((vmax-c)^2-(vmin-c)^2, x,0,1)
\end{sagesilent}

\latexProblemContent{
\ifVerboseLocation This is Volume Compute Question 0013. \\ \fi
\begin{problem}
Find the volume of the solid generated when the are bounded by the curve $y=\sage{vmin}$ and $y=\sage{vmax}$ is revolved about the line $y=\sage{c}$. (Use either washers or shells)

\[\answer{\sage{Ans}}\; \mbox{units}^3\]

\end{problem}}%}
%%%%%%%%%%%%%%%%%%%%%%

%%%%%%%%%%%%%%%%%%%%%%%
%%\tagged{Ans@ShortAns, Type@Compute, Topic@Volume, Sub@Washer, Sub@Shells, File@0014}{
\begin{sagesilent}
var('x,y')
# Define variables and constants/exponents
c=NonZeroInt(-5,5,[0,1])
p=randint(0,2)
q=NonZeroInt(0,2,[p])

#Defining function choices
v=[x^3, x^2, x]
vflip=[(y)^(1/3), (y)^(1/2), y]
vmin=v[min(p,q)]
vmax=v[max(p,q)]
Vmin=vflip[min(p,q)]
Vmax=vflip[max(p,q)]

#Define the volume
if c>1:
   Ans=pi*integral((c-Vmax)^2-(c-Vmin)^2, y,0,1)
if c<0:
   Ans=pi*integral((Vmin-c)^2-(Vmax-c)^2, y,0,1)
\end{sagesilent}

\latexProblemContent{
\ifVerboseLocation This is Volume Compute Question 0014. \\ \fi
\begin{problem}
Find the volume of the solid generated when the are bounded by the curve $y=\sage{vmin}$ and $y=\sage{vmax}$ is revolved about the line $x=\sage{c}$. (Use either washers or shells)

\[\answer{\sage{Ans}}\; \mbox{units}^3\]

\end{problem}}%}
%%%%%%%%%%%%%%%%%%%%%%